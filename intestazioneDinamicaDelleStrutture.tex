% !TeX encoding = UTF-8
\documentclass[12pt,a4paper]{article}
\usepackage[latin1]{inputenc}
\usepackage{amsmath}
\usepackage{amsfonts}
\usepackage{amssymb}
\usepackage{makeidx}
\usepackage{graphicx}
\usepackage[italian]{babel}

\title{\Huge Appunti di Teoria e Dinamica\\delle Strutture\\[1.5ex] \large Universit� degli Studi di Trento\\Dipartimento di Ingegneria Civile, Ambientale e Meccanica\\Corso di Laurea Magistrale in Ingegneria Civile}
\author{Matteo Franzoi\\matteo.franzoi-1@studenti.unitn.it}
\date{Ultimo aggiornamento: \today}


\begin{document}
	\maketitle
	\thispagestyle{empty}
	
	I seguenti appunti sono stati scritti durante il corso di Teoria e Dinamica delle Strutture tenuto dal Professore Francesco Dal Corso durante l'anno accademico 2019/2020. Gli appunti possono contenere errori; nel caso il lettore ne riscontrasse � invitato a comunicarli inviando una mail all'indirizzo sopra riportato.

\cleardoublepage
\pagestyle{empty}
\tableofcontents
\thispagestyle{empty}
\cleardoublepage
\setcounter{page}{1}\section{Introduzione}
\newpage
\setcounter{page}{5}\section{Teoria della plasticit�}
\newpage
\setcounter{page}{10}\subsection{Plasticit� 1D - legame costitutivo monodimensionale}
\newpage
\setcounter{page}{16}\subsection{Criteri di resistenza per materiali isotropi}
\newpage
\setcounter{page}{18}\subsection{Plasticit� 3D}
\newpage
\setcounter{page}{22}\subsection{Equazione di Prandtl - Reuss (Flow theory of plasticity - 1921)}
\newpage
\setcounter{page}{24}\subsection{Postulato di Drucker (o di massima dissipazione)}
\newpage
\setcounter{page}{27}\subsection{Problema al contorno}
\newpage
\setcounter{page}{28}\subsection{Collasso plastico}
\newpage
\setcounter{page}{29}\subsection{Analisi limite}
\newpage
\setcounter{page}{30}\subsubsection{Teorema zero}
\newpage
\setcounter{page}{31}\subsubsection{Teorema statico (o del limite inferiore)}
\newpage
\setcounter{page}{33}\subsubsection{Teorema cinematico (o del limite superiore)}
\newpage
\setcounter{page}{35}\subsection{Azione assiale plastica}
\newpage
\setcounter{page}{53}\subsection{Flessione plastica}
\newpage
\setcounter{page}{53}\subsubsection{Sezione a doppio asse di simmetria}
\newpage
\setcounter{page}{57}\subsubsection{Sezione rettangolare}
\newpage
\setcounter{page}{59}\subsubsection{Sezione a doppio 'T'}
\newpage
\setcounter{page}{60}\subsubsection{Sezione a un asse di simmetria}
\newpage
\setcounter{page}{60}\subsubsection{Sezione a triangolo isoscele}
\newpage
\setcounter{page}{61}\subsubsection{Sezione a 'T' sottile}
\newpage
\setcounter{page}{63}\subsubsection{Sezione a 'T' spessa}
\newpage
\setcounter{page}{64}\subsection{Cerniera plastica}
\newpage
\setcounter{page}{67}\subsubsection{Sistemi a una campata - esercizi}
\newpage
\setcounter{page}{81}\subsubsection{Sistemi a pi� campate - esercizi}
\newpage
\setcounter{page}{103}\subsection{Collasso e deformabilit�}
\newpage
\setcounter{page}{105}\subsection{Domini di interazione}
\newpage
\setcounter{page}{106}\subsubsection{Dominio M-N}
\newpage
\setcounter{page}{110}\subsubsection{Dominio M-T}
\newpage
\setcounter{page}{111}\subsection{Cerniera plastica mobile}
\newpage

%----------------------Dinamica--------------------------

\setcounter{page}{114}\section{Dinamica delle Strutture}
\newpage
\setcounter{page}{118}\subsection{Oscillatore semplice a 1 gdl}
\newpage
\setcounter{page}{121}\subsubsection{Moto impresso}
\newpage
\setcounter{page}{122}\subsubsection{Oscillazioni libere ($F(t) = 0$)}
\newpage
\setcounter{page}{125}\subsubsection{Forzante costante a gradino}
\newpage
\setcounter{page}{127}\subsubsection{Forzante sinusoidale}
\newpage
\setcounter{page}{130}\subsubsection{Oscillatore semplice smorzato (viscosamente)}
\newpage
\setcounter{page}{133}\subsubsection{I� metodo di valutazine di $\nu_{eq}$}
\newpage
\setcounter{page}{133}\subsubsection{Smorzamento per attrito (Coulomb)}
\newpage
\setcounter{page}{135}\subsubsection{Oscillazioni smorzate con forzante a gradino}
\newpage
\setcounter{page}{136}\subsubsection{Oscillazioni smorzate con forzante sinusoidale}
\newpage
\setcounter{page}{139}\subsubsection{II� metodo di valutazine di $\nu_{eq}$ (sperimentalmente)}
\newpage
\setcounter{page}{139}\subsubsection{Ciclo di isteresi dinamica}
\newpage
\setcounter{page}{141}\subsubsection{Forzante periodica}
\newpage
\setcounter{page}{141}\subsubsection{Forzante onda quadra}
\newpage
\setcounter{page}{147}\subsubsection{Vibrometro e accelerometro}
\newpage
\setcounter{page}{149}\subsubsection{Forzante generica}
\newpage
\setcounter{page}{153}\subsubsection{Forzante lineare nel tempo}
\newpage
\setcounter{page}{155}\subsubsection{Forzante a gradino con crescita lineare}
\newpage

%----------------------N gdl------------------------------

\setcounter{page}{161}\subsection{Sistemi a N gdl}
\newpage
\setcounter{page}{161}\subsubsection{Equazioni di Lagrange}
\newpage
\setcounter{page}{165}\subsubsection{Condensazione statica}
\newpage
\setcounter{page}{172}\subsubsection{Oscillazioni libere non smorzate}
\newpage
\newpage
\setcounter{page}{174}\subsubsection{Normalizzazione degli autovalori}
\newpage
\setcounter{page}{175}\subsubsection{Ortogonalit� dei modi di vibrare}
\newpage
\setcounter{page}{181}\subsubsection{Coordinate principali}
\newpage
\setcounter{page}{183}\subsubsection{Risposta forzata}
\newpage
\setcounter{page}{184}\subsubsection{Forze equivalenti a moto impresso}
\newpage
\setcounter{page}{200}\subsubsection{Smorzamento in sistemi a N gdl}
\newpage
\setcounter{page}{201}\subsubsection{Smorzamento classico}
\newpage
\setcounter{page}{201}\subsubsection{Smorzamento di Rayleigh (o semplice)}
\newpage
\setcounter{page}{203}\subsubsection{Smorzamento di Caughey}
\newpage
\setcounter{page}{204}\subsubsection{Combinazione delle azioni in dinamica}
\newpage
\setcounter{page}{223}\subsubsection{Costruzioni simmetriche e costruzioni con pianta non simmetrica}
\newpage
\setcounter{page}{230}\subsubsection{Trascurabilit� dell'inerzia torsionale propria}
\newpage
\setcounter{page}{231}\subsubsection{Analisi dinamica di aste inclinate}
\newpage
\setcounter{page}{235}\subsubsection{Principi di isolamento}
\newpage

%----------------------Sistemi continui-----------------
\setcounter{page}{238}\subsection{Sistemi continui}
\newpage
\setcounter{page}{238}\subsubsection{Vibrazioni flessionali di travi}
\newpage
\setcounter{page}{239}\subsubsection{Oscillazioni libere}
\newpage
\setcounter{page}{249}\subsubsection{Smorzamento in sistemi continui}
\newpage
\setcounter{page}{255}\subsubsection{Sistemi a pi� campate}
\newpage
\setcounter{page}{258}\subsubsection{Influenza dell'inerzia rotazionale}
\newpage
\setcounter{page}{259}\subsubsection{Influenza dell'azione assiale}
\newpage
\setcounter{page}{260}\subsubsection{Oscillazioni smorzate}
\newpage
\setcounter{page}{261}\subsubsection{Smorzamento classico}
\newpage
\setcounter{page}{263}\subsubsection{Carico concentrato con modulo dinamico }
\newpage
\setcounter{page}{265}\subsubsection{Carico mobile su ponte}
\newpage

%------------------Sistemi a 1 gdl generalizzato-------------

\setcounter{page}{267}\subsection{Sistemi a 1 gdl generalizzato}
\end{document}
